\documentclass{scrartcl}
\usepackage{mathtools}
\newcommand\E{\mathbf{E}}
\renewcommand\P{\mathbf{P}}
\newcommand\1{\mathbf{1}}
\begin{document}
Andreas Haupt, Louis Faucon

Topics in Theoretical Computer Science 

Homework II

\section{Strong Duality Theorem}

\section{Maximum Disjoint Paths}

\section{Maximum Weight Spanning Tree}

\section{k-Disjoint Perfect Matchings}

The integrality of the bipartite perfect matching polytope gives us a polynomial time algorithm $A$ to compute a perfect matching. Given this algorithm, and a k-regular bipartite graph $G = (A \cup B,E)$, we proceed as follow :

\begin{verbatim}
while k > 0
    new_matching := A(G) 
    perfect_matching_partition.add(new_matching)
    E = E\new_matching
    k--
endwhile
\end{verbatim}

The correctness of this algorithm is easily verified by realising that the retrieval of edges from a perfect matching to a k-regular bipartite graph gives a (k-1)-regular bipartite graph. It also clearly runs in polynomial time.

\end{document}