\documentclass{scrartcl}
\usepackage{mathtools}
\newcommand\E{\mathbf{E}}
\renewcommand\P{\mathbf{P}}
\newcommand\1{\mathbf{1}}
\usepackage{hyperref}
\usepackage{csquotes}
\usepackage[standard, thref]{ntheorem}
\begin{document}
Andreas Haupt, Louis Faucon

Topics in Theoretical Computer Science 

Homework II
\section{Strong Duality Theorem}
Throughout this task, Let $K$ be a ordered field, $A \in K^{m\times n}$, $b \in K^{1 \times m}$. Note that the part of the Strong Duality theorem, that unboundedness resp. infeasibility of the primal implies infeasibility resp. unboundedness of the dual and vice versa is readily proved: we do not need to prove the \enquote{vice versa} part, as this one follows from the fact, that the dual of the dual is equivalent to the original problem. If the primal 
\begin{Lemma}[Farkas]
$(\exists x \ge 0 \colon Ax=b) \Leftrightarrow (K \ni u \ge 0\wedge u^TA=0 \ge 0 \Rightarrow yb \ge 0)$
\end{Lemma}
\begin{Proof}
Let $Ax \le b$ be given. By multiplication of $A$ and $b$ with positive constants, we may assume, that the elements in $A$'s first column are $0$ or $\pm 1$, by reordering the rows of $A$ (and the elements of $b$), that it has the form
\begin{alignat*}
\forall 1 \le i \le m_1 \colon &&a_i' x'& \le b_i'\\
\forall m_1+1 \le i \le m_2 \colon &&x+a_i' x'& \le b_i'\\
\forall m_2+1 \le i \le m \colon &&-x+a_i' x'& \le b_i'
\end{alignat*}
where $a_i'$ is the row vector that contains the second to $n$th element of $A$'s $i$th row in the multiplied system. This system of equations is equivalent to the system
\begin{alignat*}
\forall 1 \le i \le m_1 \colon &&a_i' x'& \le b_i\\
\forall m_1+1 \le j \le m_2 \forall m_2+1 \le k\le m \colon &&a_j' x'-b_j& \le b_k - a_k'x',
\end{alignat*}
. Repeating this, we may reduce the dimension of $x$ to zero. Then 
\end{Proof}
\begin{Theorem}

\end{Theorem}
\begin{Proof}
We have $ \neq \emptyset $, $ \{y|y^TA = c^T \wedge y\ge 0\}$. Then
\[
\forall x \in \{x|Ax\le b\} \forall y \in  \{y|y^TA = c^T \wedge y\ge 0\} \colon c^Tx =y^TAx \le y^Tb,
\]
therefore it remains to show, that 
\[
\exists x \in \{x|Ax\le b\} \exists y \in  \{y|y^TA = c^T \wedge y\ge 0\} c^Tx\ge y^Tb
\]
It is equivalent, that
\[
\{(x,y) | A x \le b, A^Ty = c, -c^T + y^Tb \le 0, y \ge 0\}\neq \emptyset,
\]
which is, by \thref{lem:farkas}, equivalent to 
\[
\{(u,v,w) | u^TA - wc^T = 0 , v^TA^T + wb^T \ge 0, u^Tb+v^Tc < 0 , u \ge 0, w \ge 0\}= \emptyset.
\]
We suppose, this was not the case. If there was a solution of the form $(u,v,0)$
\[
%:
\]
If otherwise we had a solution of the form $(u,v,w), w >0$. Then. 
\end{Proof}

\section{Maximum Disjoint Paths}

The dual of the given linear problem is :

\paragraph{}
\begin{tabular}{ll}
Minimize & $\sum_{e \in E} y_e$\\
Subject to & $\forall p \in P, \ \sum_{e \in p} y_e \geq 1$\\
& $\forall e \in E, \  y_e \geq 0$ \\
\end{tabular}
\paragraph{} 

Indeed, if we want an upper-bound on the value of the maximization, we have to take something that includes at least once every $x_p$, which means at least one of the edge that is contained by $p$ must be taken into account. 

This duality is simply the LP version of MAXFLOW/MINCUT duality when all weights are equal to 1. A binary solution of the above dual gives a cut of the graph with a minimum number of edges.

\section{Maximum Weight Spanning Tree}





\section{k-Disjoint Perfect Matchings}

The integrality of the bipartite perfect matching polytope gives us a polynomial time algorithm $A$ to compute a perfect matching. Given this algorithm, and a $k$-regular bipartite graph $G = (A \cup B,E)$, we proceed as follow :

\begin{verbatim}
while k > 0
    new_matching := A(G) 
    perfect_matching_partition.add(new_matching)
    E = E\new_matching
    k--
endwhile
\end{verbatim}

The correctness of this algorithm is easily verified by realising that the retrieval of edges from a perfect matching to a $k$-regular bipartite graph gives a $(k-1)$-regular bipartite graph. It also clearly runs in polynomial time.

\end{document}