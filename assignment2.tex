\documentclass{scrartcl}
\usepackage{mathtools}
\newcommand\E{\mathbf{E}}
\renewcommand\P{\mathbf{P}}
\newcommand\1{\mathbf{1}}
\begin{document}
Andreas Haupt, Louis Faucon

Topics in Theoretical Computer Science 

Homework II

\section{Strong Duality Theorem}

\section{Maximum Disjoint Paths}

The dual of the given linear problem is :

\paragraph{}
\begin{tabular}{ll}
Minimize & $\sum_{e \in E} y_e$\\
Subject to & $\forall p \in P, \ \sum_{e \in p} y_e \geq 1$\\
& $\forall e \in E, \  y_e \geq 0$ \\
\end{tabular}
\paragraph{} 

Indeed, if we want an upper-bound on the value of the maximization, we have to take something that includes at least once every $x_p$, which means at least one of the edge that is contained by $p$ must be taken into account. 

This duality is simply the LP version of MAXFLOW/MINCUT duality when all weights are equal to 1. A binary solution of the above dual gives a cut of the graph with a minimum number of edges.

\section{Maximum Weight Spanning Tree}

\section{k-Disjoint Perfect Matchings}

The integrality of the bipartite perfect matching polytope gives us a polynomial time algorithm $A$ to compute a perfect matching. Given this algorithm, and a k-regular bipartite graph $G = (A \cup B,E)$, we proceed as follow :

\begin{verbatim}
while k > 0
    new_matching := A(G) 
    perfect_matching_partition.add(new_matching)
    E = E\new_matching
    k--
endwhile
\end{verbatim}

The correctness of this algorithm is easily verified by realising that the retrieval of edges from a perfect matching to a k-regular bipartite graph gives a (k-1)-regular bipartite graph. It also clearly runs in polynomial time.

\end{document}