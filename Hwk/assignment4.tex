\documentclass{scrartcl}
\usepackage{mathtools}
\newcommand\E{\mathbf{E}}
\renewcommand\P{\mathbf{P}}
\newcommand\1{\mathbf{1}}
\usepackage{hyperref}
\usepackage{csquotes}
\usepackage[standard, thref]{ntheorem}
\begin{document}
Andreas Haupt, Louis Faucon

Topics in Theoretical Computer Science 

Homework IV


\section{Perfect Matchings}
\begin{enumerate}
\item
It suffices to give a feasible point in the perfect matching polytope of this graph. A theorem by Edmonds proved in class shows, that this polytope is characterized by the inequalities
\begin{align*}
x_e &\ge 0 , \quad \forall e \in E (G) \\
x(\delta (v)) &=1, \quad \forall v \in V(G) \\
x (\delta (S)) &\ge 1, \quad \forall S \subset \binom{V}{k}, k \in 2\mathbb{Z}+1
\end{align*}
We claim, that the point $x$, $x_e = \frac{1}{3}, \forall e \in E(G)$ is feasible. The first $\lvert E(G)\rvert$ inequalities are obviously satisfied, the following $\lvert V(G)\rvert$ by $3$-regularity. For the third consider an odd subset of vertices $S$ of size at least $3$. Clearly, it suffices to show, that $\lvert \delta (S) \rvert \ge 3$. We know
\begin{align*}
\lvert\delta (S)\rvert &\ge 2
\end{align*}
by $2$-connectedness, so it suffices to show, that $\lvert \delta (S) \rvert$ is odd. By a double-counting argument, we have
\begin{align*}
\lvert \delta (S) \rvert + 2\lvert E(G[S])\rvert = 3 \lvert S \rvert
\end{align*}
where the right hand side is odd and the second summand on the left side is even. This shows that $\lvert \delta (S) \rvert$ is odd and completes the proof.
\item
\item
\end{enumerate}

\section{Ellipsoid Method}
\begin{enumerate}
\item
We first verify the $l$ inequalities. Then, if it exists, we can find $x$ s.t. $x^T M x < 0$ in polynomial time by diagonalising. Then a separator hyperplane will be $H = \{A\ |\ (M-A) x = 0\}$. 

\item


\end{enumerate}

\end{document}