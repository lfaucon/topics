\documentclass{scrartcl}
\usepackage{mathtools}
\newcommand\E{\mathbf{E}}
\renewcommand\P{\mathbf{P}}
\newcommand\1{\mathbf{1}}
\usepackage{hyperref}
\usepackage{csquotes}
\usepackage[standard, thref]{ntheorem}
\begin{document}
Andreas Haupt, Louis Faucon

Topics in Theoretical Computer Science 

Homework IV


\section{Perfect Matchings}
\begin{enumerate}
\item
It suffices to give a feasible point in the perfect matching polytope of this graph. A theorem by Edmonds proved in class shows, that this polytope is characterized by the inequalities
\begin{align*}
x_e &\ge 0 , \quad \forall e \in E (G) \\
x(\delta (v)) &=1, \quad \forall v \in V(G) \\
x (\delta (S)) &\ge 1, \quad \forall S \subset \binom{V}{k}, k \in 2\mathbb{Z}+1
\end{align*}
We claim, that the point $x$, $x_e = \frac{1}{3}, \forall e \in E(G)$ is feasible. The first $\lvert E(G)\rvert$ inequalities are obviously satisfied, the following $\lvert V(G)\rvert$ by $3$-regularity. For the third consider an odd subset of vertices $S$ of size at least $3$. Clearly, it suffices to show, that $\lvert \delta (S) \rvert \ge 3$. We know
\begin{align*}
\lvert\delta (S)\rvert &\ge 2
\end{align*}
by $2$-connectedness, so it suffices to show, that $\lvert \delta (S) \rvert$ is odd. By a double-counting argument, we have
\begin{align*}
\lvert \delta (S) \rvert + 2\lvert E(G[S])\rvert = 3 \lvert S \rvert
\end{align*}
where the right hand side is odd and the second summand on the left side is even. This shows that $\lvert \delta (S) \rvert$ is odd and completes the proof.
\item
We assign now weights to the edges of graph $G$. We give $f,g$ weight $2$ and the rest weight $1$. Then our (still feasible) solution from the first part has weight
\[
\frac{\lvert V\rvert}{2} + \frac{2}{3}
\]
Now by the integrality of the perfect matching polytope, there is a strictly lighter integral solution. As this one has to have integral weight, it has to be of weight at most $\frac{\lvert V \rvert}{2}$, which means, that there is a perfect matching containing neither $f$ nor $g$.
\item
Consider such a graph. Let $e=\{v,w\}$ be the unique bridge where $v$ has neighbors $v^1$ and $v^2$ and . Now consider a new graph that has as edge set 
\[
V = V(G) \setminus \{v,w\}\cup \{v_1, w_1, v_2, v_2\}
\]
and edge set 
\begin{multline*}
E=E(G) \setminus \{\{v,v^1\},\{v,v^2\},\{w,w^1\},\{w,w^2\}\}\\ \cup \{\{v_1,v^1\},\{v_2,v^2\},\{w_1,w^1\},\{w_2,w^2\},\{v_1,v_2\},\{w_1,w_2\}\}.
\end{multline*}
This graph then is $3$-regular and $2$-connected as one easily verifies. By the second part of the task, if we delete edges $\{v_1,v^1\}$ and $\{w_1,w^1\}$, we can find a perfect matching in the graph. This perfect matching has to contain either the edges $\{v_1,w_1\}$ and $\{v_2,w_2\}$ or $\{v_1, v_2\}$ and $\{w_1, w_2\}$. But in both cases, we then get a perfect matching for the original graph by taking the perfect matching in this graph with the exception, that we contract $\{v_1, v_2\}$ and $\{w_1, w_2\}$ calling the vertices resulting from this contraction $v$ resp. $w$ and including the edge $\{v,w\}$. 
\end{enumerate}

\section{Ellipsoid Method}
\begin{enumerate}
\item
At first, we can decide whether a semidefinite problem is feasible efficiently: We first verify the $m$ inequalities ($O(n^2m)$). We can compute the eigenvalues of the 

Then, if it exists, we can find $x$ s.t. $x^T M x < 0$ in polynomial time by diagonalising. Then a separator hyperplane will be $H = \{A\ |\ (M-A) x = 0\}$. 

\item


\end{enumerate}

\end{document}