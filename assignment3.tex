\documentclass{scrartcl}
\usepackage{mathtools}
\newcommand\E{\mathbf{E}}
\renewcommand\P{\mathbf{P}}
\newcommand\1{\mathbf{1}}
\usepackage{hyperref}
\usepackage{csquotes}
\usepackage[standard, thref]{ntheorem}
\begin{document}
Andreas Haupt, Louis Faucon

Topics in Theoretical Computer Science 

Homework III


\section{Matroid Intersection}
To see that for this problem an efficient (i.e. polynomial) algorithm exists, we note, 
\begin{itemize}
\item that by the matroid intersection theorem the maximum cardinality set that is independent in two matroids at the same can be found efficiently. 
\item that the problem is equivalent to 
\[
\max_{S \in \mathcal{I}_1\cap \mathcal{I}_2} \lvert S\rvert =\frac{1}{2} \sum_{v \in V(G)}( i(v) + o(v)),\footnote{This is a double counting argument.}
\]
where the ground set for the partition matroids $\mathcal{I}_1$ and $\mathcal{I}_2$ is $E(G)$ and
\begin{align*}
\mathcal{I}_1& \coloneqq \{ A \subseteq E(G) | \forall v \in V(G) \colon \delta_A^- (v) \le i(n)\}\\
\mathcal{I}_2& \coloneqq \{ A \subseteq E(G) | \forall v \in V(G) \colon \delta_A^+ (v) \le o(n)\}.
\end{align*}
\end{itemize}
\section{Matroids}
We first prove heredity. Let $A \subseteq B$. If $B$ is good, let $M_B$ be a matching that matches $B$. Then $M_B \cap \delta_G (A)$ matches $A$.

Second, we have to prove the exchange property. Let $V$, $W$, $\lvert W\rvert > \lvert V\rvert$  be good and let $M_V$ resp. $M_W$ be matchings that match them. For the undirected graph $G'= (\Gamma_G (A \cup B), M_V \cup M_W)$ consider its orientation $G''$ where all edges of $M_V$ resp. $M_W$ leave $A$ and enter $B$ resp. leave $B$ and enter $A$.

$G''$ has in- and out-degrees one or two, thus the graph consists of paths and circles. Since $\lvert M_W \rvert = \lvert W \rvert > \lvert V \rvert= \lvert M_V\rvert$, there is a path $P$ that begins in $v \in W\setminus V$\footnote{Since $G$ and so also $G''$ is bipartite, there are only even circles that are $M_V$-$M_W$-alternating by the choice of orientation. Deleting the circles, we remain with a union of $M_V$-$M_W$-alternating paths, which contain altogether more edges of $M_W$ than of $M_W$. Thus one path begins and ends in $M_V$, by the orientation beginning in $e \in W\setminus V$.}. But $M_V \Delta E(P)$ matches all vertices in $P$ and $e$. Thus $P \cup \{e\}$ is good.
\section{Spanning Trees}
\begin{enumerate}
\item

\item

\end{enumerate}

\end{document}