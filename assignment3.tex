\documentclass{scrartcl}
\usepackage{mathtools}
\newcommand\E{\mathbf{E}}
\renewcommand\P{\mathbf{P}}
\newcommand\1{\mathbf{1}}
\usepackage{hyperref}
\usepackage{csquotes}
\usepackage[standard, thref]{ntheorem}
\begin{document}
Andreas Haupt, Louis Faucon

Topics in Theoretical Computer Science 

Homework III


\section{Matroid Intersection}
To see that for this problem an efficient (i.e. polynomial) algorithm exists, we note, 
\begin{itemize}
\item that by the matroid intersection theorem the maximum cardinality set that is independent in two matroids at the same can be found efficiently. 
\item that the problem is equivalent to 
\[
\max_{S \in \mathcal{I}_1\cap \mathcal{I}_2} \lvert S\rvert =\frac{1}{2} \sum_{v \in V(G)}( i(v) + o(v)),\footnote{This is a double counting argument.}
\]
where the ground set for the partition matroids $\mathcal{I}_1$ and $\mathcal{I}_2$ is $E(G)$ and
\begin{align*}
\mathcal{I}_1& \coloneqq \{ A \subseteq E(G) | \forall v \in V(G) \colon \delta_A^- (v) \le i(n)\}\\
\mathcal{I}_2& \coloneqq \{ A \subseteq E(G) | \forall v \in V(G) \colon \delta_A^+ (v) \le o(n)\}.
\end{align*}
\end{itemize}
\section{Matroids}
We first prove heredity. Let $A \subseteq B$. If $B$ is good, let $M_B$ be a matching that matches $B$. Then $M_B \cap \delta_G (A)$ matches $A$.

Second, we have to prove the exchange property. Let $V$, $W$, $\lvert W\rvert > \lvert V\rvert$  be good and let $M_V$ resp. $M_W$ be matchings that match them. For the undirected graph $G'= (\Gamma_G (A \cup B), M_V \cup M_W)$ consider its orientation $G''$ where all edges of $M_V$ resp. $M_W$ leave $A$ and enter $B$ resp. leave $B$ and enter $A$.

$G''$ has in- and out-degrees one or two, thus the graph consists of paths and circles. Since $\lvert M_W \rvert = \lvert W \rvert > \lvert V \rvert= \lvert M_V\rvert$, there is a path $P$ that begins in $v \in W\setminus V$\footnote{Since $G$ and so also $G''$ is bipartite, there are only even circles that are $M_V$-$M_W$-alternating by the choice of orientation. Deleting the circles, we remain with a union of $M_V$-$M_W$-alternating paths, which contain altogether more edges of $M_W$ than of $M_W$. Thus one path begins and ends in $M_V$, by the orientation beginning in $e \in W\setminus V$.}. But $M_V \Delta E(P)$ matches all vertices in $P$ and $e$. Thus $P \cup \{e\}$ is good.
\section{Spanning Trees}
\begin{enumerate}
\item
We note, that there is a mistake in the task: The weights must be assumed to be positive, since otherwise the weight function is a counterexample for the first inequality.

It was shown in the lecture, that the spanning tree polytope is integral. Therefore, it suffices to show, that there is a fractional solution $(x_e)_{e \in E(G)}$ for the spanning tree polytope, for that we have
\[
\sum_{e \in E} x_e w(e) < \frac{2}{k} \sum_{e \in E} w(e)
\]
for the integrality will yield an integral solution with at most as much weight. 

To obtain this solution, we choose an arbitrary edge set in the graph of size $\frac{k}{2}$.\footnote{The following equation only works for the case, that $k\in 2\mathbb{N}$. Otherwise, we choose still a set of $\frac{k}{2}$ edges, and choose from this set an edge $e$ and set $x_e = \frac{1}{k}$ (letting the rest stay the same). The calculations go through the same way with this modification, as one notices.}
\[
x_e = \begin{cases} \frac{2}{k} & e \in S \\ 0 &\text{else.}\end{cases}
\]
Then:
\[
\sum_{e \in E(G)} x_e w(e) = \frac{2}{k}\sum_{e \in E(G) \setminus S} w(e) <  \frac{2}{k}\sum_{e \in E(G)} w(e)
\]
and by $k$-regularity and $k$-connectedness
\begin{align*}
\sum_{e \in E} x_e &= \frac{2}{k}\sum_{e \in E(G) \setminus S} 1 =\frac{2}{k}( \lvert E(G) \rvert - \frac{k}{2}) = \lvert V(G) \rvert -1\\
\sum_{e \in \delta (S)} x_e &\ge \frac{2}{k} (k-\frac{k}{2}) = 1
\end{align*}
The positivity is obvious.
\item
We consider the matroid intersection $(E(G), \mathcal{I}_1\cap \mathcal{I}_2)$ with $\mathcal{I}_1$ the graphic matroid and $\mathcal{I}_2$ the partition matroid whose independent edge sets are those, for whom the degrees of all vertices is restricted to two. The corresponding matroid intersection polytope is integral. Therefore, it suffices to show, that there is some feasible solution of this polytope with\footnote{we take the weight vector in a maximization over this polytope to be constantly one}
\[
\sum_{e \in E(G)} x_e  \ge 
\]
since for an integral solution of the polytope
\[
\sum_{e \in E(G)} x_e = \frac{1}{2} \sum_{v \in V(G)} x(\delta(v)) = 1 \lvert\{v \in V| \lvert\delta (v)\rvert=2\}\rvert + \frac{1}{2} \lvert\{v \in V| \lvert\delta (v)\rvert=1\}\rvert
\]
\end{enumerate}

\end{document}